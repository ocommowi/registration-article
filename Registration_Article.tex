\documentclass[a4paper]{article}
\usepackage[utf8]{inputenc}
\usepackage{geometry}

\usepackage{algorithm}
\usepackage[noend]{algorithmic}
\usepackage{amsmath,amssymb}
\usepackage{hyperref}

\usepackage[pdftex]{graphicx}
\DeclareGraphicsExtensions{.jpg,.png}
\graphicspath{{.}{images/}}

\usepackage{fancyhdr}                    % Fancy Header and Footer

%%% Fancy Header %%%%%%%%%%%%%%%%%%%%%%%%%%%%%%%%%%%%%%%%%%%%%%%%%%%%%%%%%%%%%%%%%%
% Fancy Header Style Options

\pagestyle{fancy}                       % Sets fancy header and footer
\fancyfoot{}                            % Delete current footer settings

\fancyhead[R]{\bfseries\thepage} 
\fancyhead[L]{\bfseries O. Commowick - A versatile framework for images registration}

\def\argmax{\operatornamewithlimits{arg\,max}}
\def\argmin{\operatornamewithlimits{arg\,min}} 

\title{A versatile block-matching framework for images registration}
\author{Olivier Commowick}

\begin{document}

\maketitle

\begin{abstract}
This paper presents new methodological developments and experiments, quite incremental but at the core of the block-matching registration algorithm provided in the \href{https://anima.irisa.fr}{Anima software}. The overall goal of this article is to present those advances, to include new experiments illustrating their behavior, and to provide a discussion forum on the article using the associated git repository and discussions. Updates of the paper will be posted over time when new incremental developments come into play. As of now the paper includes developments on the core of the algorithm, its symmetrization, transforms used between blocks, and global linear and non linear transformations computation.
\end{abstract}

\tableofcontents

\newpage

\section{Introduction}

The main idea of this continuous article is to be regularly updated by new developments on the registration algorithms I have been working on for years, and for which ncremental updates are not sufficient to justify proper publication. It derives from this fact that this paper is not fully and properly peer-reviewed and should be taken as is. It is however a way for me to test a new idea of evolutive papers that I have wanted to try for quite a few years now. This paper and the associated repository will evolve over time, using the possibilities offered by a Git platform to provide:

\begin{itemize}
\item releases when sufficiently important changes happen to the paper (new set of experiments, new methodological developments, etc.)
\item a discussion forum opened for discussion on the paper and methodology. It can include either reviews of a part or all of the paper, discussion on new ideas related to the paper, etc.
\end{itemize}

This article aims at performing magnetic resonance images (MRI) registration. This field is a very developed one, including many methods and developments, from rigid / affine registration up to highly local large diffeomorphic metric mapping (LDDMM) registration~\cite{Beg_IJCV_2005,Avants_Media_2008}. Many reviews have been published in this field covering different aspects of registration~\cite{Pluim_2003,Oliveira_2014,Viergever_2016}. While all this literature has explored many aspects of linear and non linear registration, the field is still widely open to research, for example to register well highly variable structures such as the gray matter, or to obtain robust and fast enough methods usable in a clinical context. This paper enters in this second type of approaches, focusing on obtaining robust results for virtually any type of images (diffusion tensors, higher order diffusion models, scalar images of different modalities, etc.), both for linear and non linear transformations.

To do so, I have based my research in this field on the well known block-matching technique. It has been introduced in the medical imaging community by~\cite{Ourselin_Miccai_2000} for rigid registration. This approach is in fact very versatile, usually requiring only the similarity measure between blocks (and a re-orientation scheme for images of oriented models such as tensors or higher order models) to be defined to perform the registration. While we have published over the years a few papers utilizing this technique for various tasks (rigid registration~\cite{commowick:inserm-00681610}, non linear registration~\cite{Commowick_Miccai_2012}, tensor images registration~\cite{suarez:inserm-00657707}, diffusion compartment models registration~\cite{commowick:inserm-01556476}, distorsion correction of EPI images~\cite{Hedouin_TMI_2017}), the core part of these algorithms and some subtleties of it have always remained under the hood in our papers. I thus explore in this paper a few novelties and the core parts of the algorithm that I think deserve their discussion. In the next first sections, the individual elements of the registration  are detailed, and then I provide the general algorithms in Section~\ref{sec:reg-algorithms}. Experiments on linear and non linear registration will then come in time to illustrate some aspects of the algorithm.

\section{Block-matching for medical images registration}

We define the registration algorithm as the one that seeks a transformation $T$ so that a floating image resampled by $T$, $F \circ T$, matches as much as possible a reference image $R$: $R \approx F \circ T$. With that goal in mind, the core part of the registration algorithm is based on an iterative framework which iterates three main steps:

\begin{itemize}
	\item Define a set of blocks, i.e. a subset of voxels - often a cube around a given point in the image, on the reference image $R$. Each block is defined by its center $x_i$: $B(x_i) \equiv B_i$
	\item Match these blocks from image $R$ to image $F$, i.e. find the best local transformation $A_i$ such that a similarity measure $S(R(x), F \circ A(x))$ over the block $B_i$ is optimal
	\item From the set of transformations obtained, compute an update global transformation $\delta T$ then used to update the current global transformation $T$
\end{itemize}

This algorithm is the block-matching core that is central to all variants that are used in this document. This core is included in an iterative algorithm (see Section~\ref{sec:reg-algorithms}), itself often included in a multi-resolution (pyramidal) framework to first get back large displacements from coarse resolutions and then smaller / finer displacements from finer resolutions, in a robust and faster manner. This core part has been used in many algorithms ranging from linear~\cite{Ourselin_Miccai_2000,commowick:inserm-00681610} to non linear, diffeomorphic registration~\cite{Commowick_Miccai_2012}, with various applications such as tensor registration~\cite{suarez:inserm-00657707}. It requires mainly the definition of a similarity metric $S$ between blocks to be used on a specific modality. On scalar images~\cite{Malandain_Neuroimage_2004}, given the relatively small size of a block (thus containing two or few different tissues), a linear relationship between intensities is usually enough and a square correlation coefficient is used to match images. Apart from the similarity, several points need to be defined at the general level to have an overview of the algorithm:

\begin{itemize}
 	\item The nature of transformations $A$ and $T$
 	\item How to go from a set of transformations $A_i$ and blocks $B_i$ to a global (linear or non linear) update transformation $\delta T$?
 	\item Ensuring (or not) a symmetric transformation
 \end{itemize}

\section{Which local transformations between blocks?}
\label{sec:local_trsfs_blocks}

Local transformations between a block in $R$ and $F$ are generally assumed to be linear for two reasons: 1- the block is usually sufficiently small for this assumption to be true, and 2- the search space over which the transformation has to be estimated needs to remain small enough for computation time reasons. A transformation $A_i$ in $\mathbb{R}^N$ is therefore represented as a $(N+1)\times(N+1)$ matrix. In the following, let us consider without loss of generality that $N=3$.

\subsection{Local translation}

The most common transformation used in block-matching is a local translation, i.e. a move of the block in the three directions. In other words, $A_i$ is represented as:

\begin{equation}
	A_i = \begin{pmatrix} I_3 & t_i \\ 0 & 1 \end{pmatrix}
\end{equation}
where $I_3$ is the 3$\times$3 identity matrix and $t_i$ is a $\mathbb{R}^3$ vector. In addition, its matrix logarithm is defined as a null matrix with a translation equal to that of $A_i$ ($t_i$). This transformation has proven to be very useful and is also the fastest to estimate. Originally, these three parameters have been interpreted as a sliding of the block along the grid of voxels of the second image. The easiest optimization method for a local block is thus a discrete grid search on the voxel grid of the floating image, which has the advantage of not requiring any interpolation (and is thus fast). Although enough to recover globally sub-voxel precision thanks to the iterations of the block-matching algorithm, such an optimization may miss fine sub-voxel displacements. Recent algorithms~\cite{commowick:inserm-00681610} have therefore studied optimization over the whole $\mathbb{R}^3$ space, which may use a gradient-based algorithm if gradient of the similarity measure is available or gradient-free optimization such as the BOBYQA algorithm~\cite{Powell09a}.

\subsection{Rigid transformation and beyond}

Among linear transformations, virtually any can be searched for between a block and an image. One particularly interesting and that we have explored in several recent works~\cite{commowick:inserm-00681610,Commowick_Miccai_2012} is the rigid transformation. It has several interesting properties that allow for its simple optimization and that will be useful to estimate global linear or non linear transformations. First of all, in homogeneous coordinates, $A_i$ is represented as follows:

\begin{equation}
	A_i = \begin{pmatrix} R_i & t_i \\ 0 & 1 \end{pmatrix}
\end{equation}
where $t_i$ is again a translation (also integrating the center of rotation), and $R_i$ is a rotation matrix i.e. $R_i R_i^T = I_3$ and $|| R_i ||_F = 1$. Interestingly, Rodrigues' formulas allow for the explicit computation of its logarithm and exponential~\cite{blanco2010tutorial}. In particular, the matrix logarithm of a rigid matrix $A$ is expressed as:

\begin{equation}
	\log(A) = \begin{pmatrix} w_{\times} & l \\ 0 & 0 \end{pmatrix}
\end{equation}
where $w_{\times}$ is the cross-product matrix of a vector of rotation angles $w = (w_0,w_1,w_2)^T$ and $l$ an arbitrary $\mathbb{R}^3$ vector. For the explicit logarithm and exponential formulas, refer to~\cite{blanco2010tutorial}. The parameterization of the transformation through the matrix exponential of $\log(A)$ is particularly interesting as it allows for 1- a clear and separate depiction of the six degrees of freedom (three scalars of $l$ and the angles in $w$) of the transformation (instead of the rigid rotation matrix where parameters are entangled), and 2- a direct expression of the rigid transformation in its Lie group structure, useful to perform transformation extrapolation.

Coming back to block-matching, one can now search for rigid transformations between a block and the floating image instead of just looking for translations. This is particularly useful for registration of images with articulated structures such as the spine~\cite{Commowick_Miccai_2012}. However, since the search space is now much larger (six variables instead of three), a brute force discrete search as in the translation case is much longer and, depending on the similarity measure optimized, a gradient-based or gradient-free optimization is more adapted.

As a final remark, we presented here the search over rigid transformations but any other transformation with well defined parameters covering its entire spectrum may be used.

\section{Global transformation extrapolation}

We now have from the previous step (actually the real block-matching step), a set of blocks $B_i$ defined on an image $R$ and corresponding linear transformations $\hat{A}_i$ best matching them onto $F$. We now briefly detail how to use these local transformations to infer the update transformation $\delta T$. This procedure depends on the nature of local transformations and the nature of the global transformation sought.

\subsection{Global linear transformations}

Let us consider that we are looking for a global transformation that is linear. Two cases are still possible. First, if the $\hat{A}_i$ are translations, it is relatively easy to estimate any linear transformation best summarizing the local displacements obtained. Particularly, Chapter 8 of \cite{Pennec_PhD_1996} explains very well how, through a linear least squares formulation, to estimate from a set of translations either a global translation, rigid or affine transformation and we refer the reader to these formulas for more details.

The second case arises when transformations $\hat{A}_i$ are more than just translations. In that case, we resort to the log-Euclidean framework for linear transformations~\cite{arsigny:inria-00616084} and thus formulate the least squares optimization directly on the matrix logarithms:

\begin{equation}
	\log(\delta T) = \argmax_{M} \sum_i || \log(\hat{A}_i) - M ||^2
\end{equation}
This directly leads to $\log(\delta T)$ being the weighted average of the input log-transformations. For both cases, it is important to note that block-matching is not exempt from outliers which may degrade the obtained transformation. Numerous options may be applied to deal with this problem including weighting the individual $\hat{A}_i$ by the optimum similarity measure value they obtained, or performing robust estimation such as least trimmed squares or M-estimation~\cite{Rousseeuw_Book_1987}.

\subsection{Non linear transformations}
\label{sub:non-linear-extrapolation}

When non linear transformations are sought, the extrapolation step is more difficult since many parameters come into play i.e. one 3D vector per voxel. In this section we will consider diffeomorphisms defined by their SVF as the final transformation being sought. While encompassing a reduced set of the diffeomorphisms that may be encountered (contrarily to LDDMM~\cite{Beg_IJCV_2005}), they have interesting properties that will be heavily used in this section, and that were well defined in the log-Euclidean frameworks for diffeomorphisms~\cite{arsigny:inria-00615594} and polyaffine transformations~\cite{arsigny:inria-00616084}.

In particular, we consider that $\delta T$ is defined by its SVF $\delta S$ i.e. a vector field whose exponential is $\delta T$: $\delta T = \exp(\delta S)$. The goal is to extrapolate the ``log-vector'' at each voxel of $\delta S$ from the sparse set of optimal block transformations $\hat{A}_i$ located at their block centers $x_i$. For this task, we will heavily use the matrix logarithm of linear transformations, which is explicitly defined for translations and rigid transformations.

As a side note to this class of transformations, it has been demonstrated that computing the SVF from a transformation $T$ is computationally expensive~\cite{arsigny:inria-00615594}. In addition, SVF having nice properties for statistics computation, it is desirable to always keep $T$ implicit and instead compute the SVF $S$. Transposing the transformation composition in that space however requires to use the BCH approximation~\cite{Bossa_Miccai_2007,Vercauteren_Miccai_2008}.

\subsubsection{Gaussian extrapolation} % (fold)
\label{ssub:gaussian_extrapolation}

The first and simplest way to extrapolate a dense SVF is to use Gaussian extrapolation~\cite{commowick:tel-00133432,garcia:inria-00616148}. From the set of $\log(\hat{A}_i)$ and transformations locations $x_i$, we first construct a sparse field $C$ where each voxel corresponding to $x_i$ is affected by the displacement generated by $\log(\hat{A}_i)$: $C(x_i) = \log(\hat{A}_i) x_i$. The Gaussian extrapolation then builds a dense SVF $\delta S$ from $C$ and a sparse field $W$ of weights $w_i$ attributed to each pairing (for example the optimal value of the similarity measure): $W(x_i) = w_i$. The extrapolated $\delta S$ is then defined as:

\begin{equation}
	\delta S = \frac{G_\sigma * W C}{G_\sigma * W}
\end{equation}
where $\sigma$ is the standard deviation of the Gaussian extrapolation kernel $G_\sigma$. This extrapolation works perfectly in a region where enough input matchings are present, e.g. inside the brain. On the contrary, in regions far away from the blocks, this extrapolation is meaningless and may lead to artificially large deformations. To counter this effect, an additional post-processing is performed on the obtained SVF: when far enough from any matching ($G_\sigma * W$ below a certain threshold), $\delta S$ is gradually set to identity (i.e. zero velocity field).

As for linear transformation computation, outliers in pairings need to be accounted for. A simple operation to do so is to compute at voxel locations $x_i$ the norm of the difference between $\delta S(x_i)$ and $C(x_i)$: $r_i = || C(x_i) - \delta S(x_i) ||$. From these, the mean displacement difference $r$ over the whole image is computed as well as the variance $\sigma_r^2$:

\begin{equation}
\left\{
  \begin{array}{ccl}
	r &= & \frac{1}{N} \sum_i r_i \\
	\sigma_r^2 & = & \frac{1}{N-1} \sum_i (r_i - r)^2
	\end{array}
\right.
\end{equation}
We can reject pairings from $C$ for which the residual $r_i > r + \alpha \sigma_r$ and recompute from it an outlier free $\delta S$.

% subsubsection gaussian_extrapolation (end)

\subsubsection{M-Smoothing extrapolation} % (fold)
\label{ssub:m_smoothing_extrapolation}

Although simple, the Gaussian extrapolation does not completely incorporate outlier rejection in its process in the sense that one would need to iterate over the rejection process to do so. We have therefore introduced in~\cite{Commowick_Miccai_2012} an extrapolation approach similar to the M-smoothing filter proposed by~\cite{Mrazek_Chapter_2006}. This approach looks for the best log-transformation $\log(S_k)$ at each voxel of $\delta S$ by minimizing the following criterion:

\begin{equation}
	\left( \log S_1,\ldots,\log S_n \right) = \argmin_{\log S_1,\ldots,\log S_n} \left[ \sum_{k=1}^n \sum_{i \in V_k} w_i \rho\left( || \log S_k - \log A_i||^2 \right) d\left( |x_k - x_i|^2 \right) \right]
\end{equation}
where $\rho$ is robust error norm (typically linked to an M-estimator, here the Welsch function), $V_k$ is a neighborhood around voxel $i$ (note that the sum over $i \in V_k$ only considers voxels where a transformation $A_i$ was estimated) and $d$ is a spatial error norm. This criterion can be minimized using gradient descent which for a particular adaptive, data-dependent step size leads to the following update formula for each $\log S_k$:

\begin{equation}
	\log S_k^{t+1} = \frac{\sum_{i \in V_k} w_i \rho'\left( || \log S_k - \log A_i||^2 \right)d\left( |x_k - x_i|^2 \right) \log A_i}{\sum_{i \in V_k} w_i \rho'\left( || \log S_k - \log A_i||^2 \right)d\left( |x_k - x_i|^2 \right)}
\end{equation}
where $\rho'$ acts as a tonal kernel, which for the Welsch function $\rho$ is written as $\rho'(a^2) = \exp\left(-a^2 / 2\lambda^2\right)$, and $d$ acts as a spatial kernel, here a Gaussian kernel: $d(a^2) = \exp\left(- a^2 / 2\sigma^2\right)$. The gradient descent is initialized with $\rho'(a^2) = 1$. These two kernels account simultaneously for the spatial proximity of $A_i$ and its local agreement with other local transformations. From the obtained $\log S_k$, we finally obtain $\delta S$ at each voxel by $\delta S(x_k) = \log S_k x_k$.

% subsubsection m_smoothing_extrapolation (end)

\section{Asymmetric or symmetric registration}
\label{sec:reg-algorithms}

From the previous sections, we now have all the necessary elements (apart from a few such as being able to resample images or the similarity measure which is not the topic here) to perform the registration of two images. The final step is to combine all of these into an algorithm. At this stage, it is crucial to note that the block-matching core algorithm is intrinsically an asymmetric one: images $R$ and $F$ do not play the same role and reverting their use does not lead to the exact inverse of $\delta T$. Options are however available to ensure this, and this is why we detail three algorithms, going from no symmetry to more and more symmetry.

\subsection{Asymmetric registration} % (fold)
\label{sub:asymmetric_registration}

The first, classical, registration built around the block-matching is the asymmetric one. It is described in Algorithm~\ref{algo:asymReg}.

\begin{algorithm}[!htbp]
\caption{Asymmetric Block-Matching Registration Algorithm}
\label{algo:asymReg}
\begin{algorithmic}[1]
  \FOR{$p=1...P$, iteration on pyramid levels,}
    \FOR{$l=1...L$, iterations,}
      \STATE{Resample $F$ with $T$}
      \STATE{Match $R$ and $F \circ T$: $\delta T \leftarrow \text{block-match} (R,F \circ T)$}
      \STATE{Update $T$ by composing it with $\delta T$}
      \STATE{If needed, regularize $T$ (elastic-like)}
    \ENDFOR
  \ENDFOR
\end{algorithmic}
\end{algorithm}
From two images, a reference $R$ and a floating image $F$, the algorithm seeks $T$ by running a multi-resolution pyramid. At each step, the previously described components are put together to estimate update $\delta T$ (or $\delta S$ if the transformation computed is non linear) and compose it with the current estimate of $T$ (BCH approximation for SVF). In this case, $R$ and $F$ clearly have an asymmetric role, blocks being always defined on $R$ and only $F$ being resampled.

% subsection asymmetric_registration (end)

\subsection{Symmetric registration} % (fold)
\label{sub:symmetric_registration}

In the previous algorithm, blocks are always defined on $R$ while $F$ is always the floating image. This second algorithm, presented in Algorithm~\ref{algo:symReg}, aims at symmetrizing this definition of blocks and ensuring that the obtained transformation when inverting $F$ and $R$ roles is the same up to an inverse, hereafter called inverse symmetry.

\begin{algorithm}[!htbp]
\caption{Symmetric Block-Matching Registration Algorithm}
\label{algo:symReg}
\begin{algorithmic}[1]
  \FOR{$p=1...P$, iteration on pyramid levels,}
    \FOR{$l=1...L$, iterations,}
      \STATE{Resample $F$ with $T$ and $R$ with $T^{-1}$}
      \STATE{Match $R$ and $F \circ T$: $\delta T_F \leftarrow \text{block-match} (R,F \circ T)$}
      \STATE{Match $F$ and $R \circ T^{-1}$: $\delta T_R \leftarrow \text{block-match} (F,R \circ T^{-1})$}
      \STATE{Compute the transform update $\delta T$ from $\delta T_F$ and $\delta T_R$}
      \STATE{Update $T$ by composing it with $\delta T$}
      \STATE{If needed, regularize $T$ (elastic-like)}
    \ENDFOR
  \ENDFOR
\end{algorithmic}
\end{algorithm}
Again, $T$ and $\delta T$ are replaced by $S$ and $\delta S$ when dealing with non linear transformations. In this algorithm, blocks are defined at each step both on $R$ and on $F$ and used to estimate two asymmetric updates: $\delta T_F$ and $\delta T_R$ (respectively $\delta S_F$ and $\delta S_R$ for non linear transformations). To account for these two updates and ensure inverse symmetry, the composition step is modified and preceded by an averaging of the two updates (for linear transformations using the matrix logarithm, and for SVF the log-Euclidean framework):

\begin{eqnarray}
	\delta T & = & \exp\left(\frac{1}{2} \left[\log(\delta T_R) - \log(\delta T_F)\right] \right) \\
	\delta S & = & \frac{1}{2} \left(\delta S_R - \delta S_F\right)
\end{eqnarray}

% subsection symmetric_registration (end)

\subsection{Kissing symmetric registration} % (fold)
\label{sub:kissing_symmetric_registration}

In direct symmetry, the images roles are not completely symmetric. In fact, one image in each way (from $F \circ T$ to $R$ and from $R \circ T^{-1}$ to $F$) is never resampled: the transformation is applied only to one image at a time. Kissing symmetry instead seeks the transformation $T$ so that $R \circ T^{-1}$ and $F \circ T$ match. $T$ now encodes the half transformation between the images: this amounts to looking for an intermediate position in between the two images by moving both of them towards each other, thereby fully symmetrizing their roles. This registration is presented in Algorithm~\ref{algo:kSymReg}.

\begin{algorithm}[!htbp]
\caption{Kissing Symmetric Block-Matching Registration Algorithm}
\label{algo:kSymReg}
\begin{algorithmic}[1]
  \FOR{$p=1...P$, iteration on pyramid levels,}
    \FOR{$l=1...L$, iterations,}
      \STATE{Resample $F$ with $T$ and $R$ with $T^{-1}$}
      \STATE{Match $R \circ T^{-1}$ and $F \circ T$: $\delta T_F \leftarrow \text{block-match} (R \circ T^{-1},F \circ T)$}
      \STATE{Match $F \circ T$ and $R \circ T^{-1}$: $\delta T_R \leftarrow \text{block-match} (F \circ T,R \circ T^{-1})$}
      \STATE{Compute the half transform update $\delta T$ from $\delta T_F$ and $\delta T_R$}
      \STATE{Update $T$ by composing it with $\delta T$}
      \STATE{If needed, regularize $T$ (elastic-like)}
    \ENDFOR
  \ENDFOR
\end{algorithmic}
\end{algorithm}
As for direct symmetry in Algorithm~\ref{algo:symReg}, the composition step is modified to compute $\delta T$, respectively $\delta S$, from the asymmetric updates:

\begin{eqnarray}
	\delta T & = & \exp\left(\frac{1}{4} \left[\log(\delta T_R) - \log(\delta T_F)\right] \right) \\
	\delta S & = & \frac{1}{4} \left(\delta S_R - \delta S_F\right)
\end{eqnarray}
The only difference with direct symmetry is here $1/4$ instead of $1/2$ to account for the fact that we are looking for a transformation bringing the two images on a middle point where they match. Applying the final transformation $T$ to $F$, is as simple as taking the square transformation (or multiply it by 2 in the ``log-space'').

\newpage 

\bibliographystyle{plain}
\bibliography{Registration_Article}


\end{document}
